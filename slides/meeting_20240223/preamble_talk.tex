%!TEX root = talk.tex

% serif fonts for mathematics
\usefonttheme{professionalfonts}

\usepackage{textcase,marvosym,cancel,booktabs,colortbl,nicefrac,ifsym}
\usepackage{fourier-orns}
\usepackage{amsmath,graphicx,svg,tikz,pgfplots,colonequals,mathtools,subfig,wrapfig,standalone,scalefnt} % MnSymbol
\usepackage{booktabs,multirow,multicol}
\usepackage{pifont} % for the dingbats in the line search plots
% \usepackage{stix} % for \sumint
% \usepackage{newtxsf}
% \usepackage{arev}
\usepackage{transparent}


\usepackage{multido}


%% not used with xetex
\usepackage{microtype}
\usepackage[utf8]{inputenc}

% a few colours, for general use.
\definecolor{lred}{RGB}{200,0,0}
\definecolor{dred}{RGB}{130,0,0}
\definecolor{dblu}{RGB}{0,0,130}
\definecolor{dgre}{RGB}{0,130,0}
\definecolor{dgra}{RGB}{50,50,50}
\definecolor{mgra}{RGB}{221,222,214}
\definecolor{lgra}{RGB}{238,238,234}
\definecolor{MPG}{RGB}{000,125,122}
\definecolor{lMPG}{RGB}{000,190,189}
\definecolor{ora}{HTML}{FF9933} %EI orange
\definecolor{lblu}{HTML}{7DA7D9}%PS blue

% Color scheme of the Vector Institute
\definecolor{Vblue}{RGB}{9,41,74}
\definecolor{Vmagenta}{RGB}{235,0,140}

\definecolor{ERC_ora}{RGB}{233,93,15}

\setlength{\parindent}{0pt}

\newcommand{\blu}[1]{{\color{dblu} #1}}   % highlight command 1
\newcommand{\ora}[1]{{\color{ora} #1}}   % highlight command 1
\newcommand{\gra}[1]{{\color{mgra} #1}}   % highlight command 1
\setbeamercolor{alerted text}{fg = Vmagenta} % highlight command 2

\setbeamercolor{normal text}{fg=black,bg=white}
\setbeamercolor{structure}{fg=Vblue}

\setbeamercolor{item projected}{use=item,fg=black,bg = Vblue}

\setbeamercolor*{palette primary}{fg=white,bg=Vblue}
\setbeamercolor*{palette secondary}{parent=palette primary,use=palette
primary,bg=dblu}
\setbeamercolor*{palette tertiary}{parent=palette
primary,use=palette primary,fg=white,bg=dgre}
\setbeamercolor*{palette
quaternary}{parent=palette primary,use=palette primary,bg=dgre}

\setbeamercolor*{block body}{bg=Vblue!15!white, fg=black}
\setbeamercolor*{block title}{parent=structure, bg=Vblue, fg=white}
\setbeamercolor{block body alerted}{bg=Gray,fg=Vblue}
\setbeamercolor{block title example}{fg=Vblue,bg=white}
\setbeamercolor{block body example}{fg=black,bg=white}

\setbeamercolor{frametitle}{bg=white,fg=Vblue}
\setbeamercolor{frametitle right}{bg=Vblue}
\setbeamercolor{framesubtitle}{fg=Vmagenta}

\setbeamerfont{framesubtitle}{size*={9}{12}}

\setbeamercolor{title}{fg=Vblue}
\setbeamercolor{subtitle}{fg=black}
\setbeamercolor{author}{fg=Black}
\setbeamercolor{date}{fg=Black}

\setbeamercolor*{titlelike}{parent=structure}

\setbeamertemplate{navigation symbols}{}
\setbeamertemplate{bibliography item}[triangle]

\setbeamerfont{frametitle}{}

% %%%% REDEFINE \emph to COLORS.

\setbeamertemplate{footline}{\hfill\color{Black}{\insertframenumber}\hspace{2ex}\null\newline\vspace{2mm}}

\arrayrulecolor{Black}

\newcommand{\filltotal}{\hspace{0pt plus 1 filll}}

\newcommand{\titlemark}[1]{
  \begin{tikzpicture}[remember picture, overlay]
    \node[draw=none,text=Gray,anchor=north east,yshift=-.8259cm] at (current
    page.north east) {\footnotesize{#1}};
  \end{tikzpicture}
}

\AtBeginEnvironment{figure}{\setcounter{subfigure}{0}}
\AtBeginEnvironment{frame}{\setcounter{footnote}{0}}

\let\oldfootnotesize\footnotesize
\renewcommand*{\footnotesize}{\oldfootnotesize\fontsize{7pt}{8pt}\selectfont}
\renewcommand\footnoterule{}

\newcommand{\graybox}[1]{
  \tikzexternaldisable%
  \begin{center}%
    \tikz{\node[fill=Black,text width=\textwidth]{%
        \begin{minipage}{1.0\linewidth}%
        \color{white}
          \begin{center}%
            #1
          \end{center}%
        \end{minipage}};}%
  \end{center}%
  \tikzexternalenable
}

\newcommand{\divider}{\noindent\makebox[\linewidth]{\rule{\paperwidth}{.4pt}}  }

\newcommand{\paperwhite}{\includegraphics[height=.6\baselineskip]{../text-document_white.png}\hspace{.5em}}
\newcommand{\paperblack}{\includegraphics[height=.6\baselineskip]{../text-document.png}\hspace{.5em}}

\newcommand{\bookwhite}{\includegraphics[height=.6\baselineskip]{../book-white.png}\hspace{.5em}}
\newcommand{\bookblack}{\includegraphics[height=.6\baselineskip]{../book-black.png}\hspace{.5em}}

\setbeamercolor{ribboncolor}{fg=black,bg=Yellow!40}
\newcommand{\ribbon}[1]{
    \begin{beamercolorbox}[wd=\paperwidth,colsep*=.3em,center]{ribboncolor}
    \setbeamertemplate{itemize items}{\color{black}\starredbullet}
    \setbeamercolor{structure}{fg=white}
    \begin{minipage}{1.0\textwidth}%
            #1
        \end{minipage}
    \end{beamercolorbox}
}
\setbeamercolor{whiteribboncolor}{fg=black,bg=white}
\newcommand{\whiteribbon}[1]{
    \begin{beamercolorbox}[wd=\paperwidth,colsep*=.5em,center,text width=\textwidth]{whiteribboncolor}
    \setbeamertemplate{itemize items}{\color{black}\starredbullet}
    \setbeamercolor{structure}{fg=black}
    \begin{minipage}{\textwidth}%
            #1
        \end{minipage}
    \end{beamercolorbox}
}

\newcommand{\bfa}[1]{
  \begin{beamercolorbox}[wd=\paperwidth,colsep*=.5em,center]{white}
    \setbeamertemplate{itemize items}{\color{black}\starredbullet}
    \tikzexternaldisable
    \begin{tikzpicture}
    \node[signal,minimum width=\paperwidth,draw=Yellow,fill=Yellow,text=black,text width=\textwidth]{
    \begin{minipage}{\textwidth}%
            #1
        \end{minipage}};
    \end{tikzpicture}
    \tikzexternalenable
    \end{beamercolorbox}
}

\newcommand{\bfi}[1]{
  \begin{beamercolorbox}[wd=\paperwidth,colsep*=.5em,center,text width=\textwidth]{white}
    \setbeamertemplate{itemize items}{\color{black}\starredbullet}
    \tikzexternaldisable
    \begin{tikzpicture}
    \node[signal,signal from=west,signal to=nowhere,minimum width=\linewidth,draw=Yellow,fill=Yellow,text=black,signal pointer angle=140]{
    \begin{minipage}{\textwidth}%
            #1
        \end{minipage}};
    \end{tikzpicture}\hspace{2.5mm}
    \tikzexternalenable
    \end{beamercolorbox}
}

\newcommand{\blackslide}{
{\setbeamercolor{background canvas}{bg=black}
  \begin{frame}[plain]
    \null
  \end{frame}
}}


\newcommand{\blackslidetext}[1]{
  {
    \setbeamercolor{frametitle}{bg=Vblue,fg=white}
    \setbeamercolor{background canvas}{bg=Vblue}%
    \setbeamercolor{structure}{fg=white}%
    \setbeamercolor{normal text}{fg=white}%
    \setbeamercolor{body}{fg=white}%
    \setbeamertemplate{itemize items}{\color{white}\starredbullet}%
    \begin{frame}
      \color{white} #1
    \end{frame}
  }
}

% logo in the upper right corner
\usepackage{eso-pic}
\newcommand\AtPagemyUpperLeft[1]{\AtPageLowerLeft{%
\put(\LenToUnit{0.83\paperwidth},\LenToUnit{0.9\paperheight}){#1}}}

%%%%%%%% TiKZ %%%%%%%%
\usetikzlibrary{arrows,shapes,plotmarks,decorations.pathmorphing}
\usetikzlibrary{backgrounds,calc,positioning,fadings}

\tikzset{>=stealth'}
\tikzstyle{graphnode} =
   [circle,draw=black,minimum size=22pt,text centered,text
     width=22pt,inner sep=0pt]
\tikzstyle{var}   =[graphnode,fill=white]
\tikzstyle{obs}   =[graphnode,fill=black,text=white]
\tikzstyle{act}   =[rectangle,draw=black,text=white,minimum
size=22pt,text centered, text width=22pt,inner sep=0pt]
\tikzstyle{fac}   =[rectangle,draw=black,fill=black!25,minimum size=5pt]
\tikzstyle{facprior} =[rectangle,draw=black,fill=black,text=white,minimum size=5pt]
\tikzstyle{edge}  =[draw=white,double=black,thick,-]
\tikzstyle{prior} =[rectangle, draw=black, fill=black, minimum size=
5pt, inner sep=0pt]
\tikzstyle{dirprior} = [circle, draw=black, fill=black, minimum
size=5pt, inner sep=0pt]

\tikzfading[name=fade top,bottom color=transparent!0,top color=transparent!75]

% to avoid warnings, copy only two symbols from stmaryrd
\DeclareSymbolFont{stmry}{U}{stmry}{m}{n}
\DeclareMathSymbol\leftarrowtriangle\mathrel{stmry}{"5E}
\DeclareMathSymbol\rightarrowtriangle\mathrel{stmry}{"5F}
\DeclareMathSymbol\sslash\mathrel{stmry}{"0C}
\DeclareMathSymbol\obar\mathrel{stmry}{"3A}
\DeclareMathSymbol\otimes\mathrel{stmry}{"0F}
\DeclareMathSymbol\ominus\mathrel{stmry}{"17}
\DeclareMathSymbol\minuso\mathrel{stmry}{"0A}
\renewcommand{\gets}{\operatorname*{\leftarrowtriangle}}
\renewcommand{\to}{\operatorname*{\rightarrowtriangle}}

\usetikzlibrary{arrows,shapes,plotmarks,pgfplots.colormaps}
\usetikzlibrary{pgfplots.groupplots}
\pgfplotsset{compat=newest}
\pgfplotsset{
  every axis legend/.append style =
    {
      cells = { anchor = east },
      draw  = none
    },
}
\makeatletter
\pgfplotsset{
    range frame/.style={
        tick align = outside,
        axis line style={opacity=0},
        after end axis/.code={
            \draw ({rel axis cs:0,0}-|{axis cs:\pgfplots@data@xmin,0}) -- ({rel axis cs:0,0}-|{axis cs:\pgfplots@data@xmax,0});
            \draw ({rel axis cs:0,0}|-{axis cs:0,\pgfplots@data@ymin}) -- ({rel axis cs:0,0}|-{axis cs:0,\pgfplots@data@ymax});
        }
    }
}
\makeatother

\pgfkeys{/pgfplots/mystyle/.style={
  % semithick,
  % tick style={major tick length=4pt,semithick,gray},
  xtick align = inside,
  ytick align = inside
  }}

\pgfkeys{/pgfplots/mytuftestyle/.style={
  semithick,
  tick style={major tick length=4pt,semithick,black},
  separate axis lines,
  axis x line*=bottom,
  axis x line shift=5pt,
  xlabel shift=0pt,
  axis y line*=left,
  tick align = outside,
  axis y line shift=5pt,
  ylabel shift=0pt}}


%%%% THEOREM environments
% \newtheorem{defi}{Definition}[section]
% \newtheorem{theo}[defi]{Theorem}
% \newtheorem{cor}[defi]{Corollary}
% \newtheorem{lemm}[defi]{Lemma}
% \newtheorem{prob}[defi]{Problem}

\newtheorem{proposition}{Proposition}[section]
% \newtheorem{lemma}[proposition]{Lemma}
% \newtheorem{theorem}[proposition]{Theorem}
% \newtheorem{corollary}[proposition]{Corollary}

\newlength{\figureheight}
\newlength{\figurewidth}
\newlength{\figheight}
\newlength{\figwidth}

% algorithms
\usepackage{algorithm}
\usepackage{algpseudocode}
\algrenewcommand{\algorithmiccomment}[1]{\hfill {$\sslash$ \scriptsize #1}}
\algrenewcommand\alglinenumber[1]{\tiny #1}
\renewcommand{\algorithmicrequire}{\textbf{Input:}}
\renewcommand{\algorithmicensure}{\textbf{Output:}}
% \algrenewcomment[1]{\hfill {\color{OliveGreen!75} \(\triangleright\) #1}}
% \algnewcommand{\LineComment}[1]{\State \(\triangleright\) #1}


% %%%%%% adding vertical bars to algorithmicx. Based on a response at
% % http://tex.stackexchange.com/questions/52473/is-it-possible-to-have-connecting-loop-lines-like-algorithm2e-in-algorithmic/52778#52778
\makeatletter
% This is the vertical rule that is inserted
\def\therule{\makebox[\algorithmicindent][l]{\hspace*{.5em}\vrule height .75\baselineskip depth .25\baselineskip}}%

\newtoks\therules% Contains rules
\therules={}% Start with empty token list
\def\appendto#1#2{\expandafter#1\expandafter{\the#1#2}}% Append to token list
\def\gobblefirst#1{% Remove (first) from token list
  #1\expandafter\expandafter\expandafter{\expandafter\@gobble\the#1}}%
\def\LState{\State\unskip\the\therules}% New line-state
\def\pushindent{\appendto\therules\therule}%
\def\popindent{\gobblefirst\therules}%
\def\printindent{\unskip\the\therules}%
\def\printandpush{\printindent\pushindent}%
\def\popandprint{\popindent\printindent}%

%      ***      DECLARED LOOPS      ***
% (from algpseudocode.sty)
\algdef{SE}[WHILE]{While}{EndWhile}[1]
  {\printandpush\algorithmicwhile\ #1\ \algorithmicdo}
  {\popandprint\algorithmicend\ \algorithmicwhile}%
\algdef{SE}[FOR]{For}{EndFor}[1]
  {\printandpush\algorithmicfor\ #1\ \algorithmicdo}
  {\popandprint\algorithmicend\ \algorithmicfor}%
\algdef{S}[FOR]{ForAll}[1]
  {\printindent\algorithmicforall\ #1\ \algorithmicdo}%
\algdef{SE}[LOOP]{Loop}{EndLoop}
  {\printandpush\algorithmicloop}
  {\popandprint\algorithmicend\ \algorithmicloop}%
\algdef{SE}[REPEAT]{Repeat}{Until}
  {\printandpush\algorithmicrepeat}[1]
  {\popandprint\algorithmicuntil\ #1}%
\algdef{SE}[IF]{If}{EndIf}[1]
  {\printandpush\algorithmicif\ #1\ \algorithmicthen}
  {\popandprint\algorithmicend\ \algorithmicif}%
\algdef{C}[IF]{IF}{ElsIf}[1]
  {\popandprint\pushindent\algorithmicelse\ \algorithmicif\ #1\ \algorithmicthen}%
\algdef{Ce}[ELSE]{IF}{Else}{EndIf}
  {\popandprint\pushindent\algorithmicelse}%
\algdef{SE}[PROCEDURE]{Procedure}{EndProcedure}[2]
   {\printandpush\algorithmicprocedure\ \textproc{#1}\ifthenelse{\equal{#2}{}}{}{(#2)}}%
   {\popandprint\algorithmicend\ \algorithmicprocedure}%
\algdef{SE}[FUNCTION]{Function}{EndFunction}[2]
   {\printandpush\algorithmicfunction\ \textproc{#1}\ifthenelse{\equal{#2}{}}{}{(#2)}}%
   {\popandprint\algorithmicend\ \algorithmicfunction}%
\makeatother

%%%% fiddling with lists:
% \usepackage{enumitem}
% try: \begin{description}[leftmargin=\parindent,labelindent=\parindent] ...

%%%% FONTS

% Monospaced
\usepackage{inconsolata}
% \usepackage[default]{lato}

% Vector's Karbon font
% \usepackage[T1]{fontenc}
% \setmainfont[Ligatures=TeX]{Karbon}

% special \vneq that works with sfmath
\makeatletter
\newcommand*{\vneq}{%
  \mathrel{%
    \mathpalette\@vneq{=}%
  }%
}
\newcommand*{\@vneq}[2]{%
  % #1: math style (\displaystyle, \textstyle, ...)
  % #2: symbol (=, ...)
  \sbox0{\raisebox{\depth}{$#1\neq$}}%
  \sbox2{\raisebox{\depth}{$#1|\m@th$}}%
  \ifdim\ht2>\ht0 %
    \sbox2{\resizebox{\vneqxscale\width}{\vneqyscale\ht0}{\unhbox2}}%
  \fi
  \sbox2{$\m@th#1\vcenter{\copy2}$}%
  \ooalign{%
    \hfil\phantom{\copy2}\hfil\cr
    \hfil$#1#2\m@th$\hfil\cr
    \hfil\copy2\hfil\cr
  }%
}
\newcommand*{\vneqxscale}{1}
\newcommand*{\vneqyscale}{.8}
\makeatother

\renewcommand{\to}{\rightarrow}


% \usepackage{palatino}
% \usepackage{mathpazo}


\let\OLDitemize\itemize
\renewcommand\itemize{\OLDitemize\addtolength{\itemsep}{0.25em}}
